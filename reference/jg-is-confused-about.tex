% -*- TeX-engine: luatex; fill-column: 72; eval: (auto-fill-mode -1); eval: (visual-fill-column-mode 1); eval: (visual-line-mode 1); eval: (adaptive-wrap-prefix-mode 1) -*-
\documentclass{article}
\usepackage{ttn}
\usepackage{booktabs}
\usepackage{mathtools}
\usepackage{colonequals} % because unicode-math has broken "::=" somehow
\usepackage{simplebnf}
\SetBNFConfig{
  relation-sym-map = { 
    {::=} = {\ensuremath{\coloncolonequals}}
  }
}
%%
%\usepackage[raggedrightboxes]{ragged2e}
%\usepackage[Ragged, size=footnote, shape=up]{sidenotesplus}
\usepackage[Ragged, size=footnote, shape=up]{sidenotesplus}
\usepackage[dvipsnames]{xcolor}
%% 
%%\usepackage[style=authoryear]{biblatex}
%%\addbibresource{notes.bib}
%%
%%
%% \defn takes an optional star and an optional argument.
%% If the optional argument is present, it is used for the marginal note; if the argument is missing, the default is the mandatory argument. if the command is starred, no marginal note is made. 
\NewDocumentCommand\defn{sO{#3}m}{\textit{#3}\IfBooleanF{#1}{\sidenote*{\textcolor{MidnightBlue}{#2}}}}
%\newcommand{\defn}[2][#2]{\textit{#2}\sidenote*{#1}}
\newcommand{\isdef}{\stackrel{\text{def}}{=}}
% Relies on euler-math.sty, loaded by ttn.sty
\newcommand{\wff}[1]{\ensuremath{\symscr{#1}}} 
%% 
\title{Things JG is confused about}
\author{James Geddes}
\date{\today}
%%
\begin{document}%\maketitle
\maketitle
\section*{Meanings of things}

\subsection*{Sentence letters}
In a sentence such as `$(P\wedge Q)$,' what is the intended
interpretation of the sentence letter~$P$?  There seem to me to be
\emph{two} possibilities:
\begin{enumerate}
\item\label{item:shorthand} it is merely a shorthand for a specific,
  previously written, English sentence;
\item it indicates an arbitrary atomic sentence.
\end{enumerate}
As evidence for~\ref{item:shorthand}, note that the book says, in
\textsection4.3, ``We will use sentence letters to represent [\dots] certain
English sentences. [\dots] for the time being, we will think of the
sentence letter of TFL, `$A$', as symbolizing the English sentence `It
is raining outside'.''  That seems pretty clear.  On the other hand,
the usual name for these things is \emph{propositional variable}.
When one writes, for example, $(P \wedge Q)$, must one already be committed
to a specific expansion of $P$ and $Q$ (as sentences of English, say)?
Or is the intended reading that one can replace $P$ and $Q$ by
whatever one likes?  I don't think it matters but the book is normally
a stickler for precision.

\subsection*{Propositional sentences}
What in fact is a sentence?  There seem to me to be two possibilities:
\begin{enumerate}
\item It is a \emph{string}: a sequence of characters.  The sentence `$P\wedge Q$' is just that sequence of characters, namely: `$P$', `$\wedge$', `$Q$';
\item It is a \emph{tree}.  The sentence `$P\wedge Q$' means ``a
  conjunction of $P$ and $Q$,'' or more concretely, perhaps, ``a tree
  with a node labelled by `$\wedge$' having two children, labelled by
  $P$ and $Q$ respectively.''
\end{enumerate}
Most texts I have read (including the book) give the impression that
they mean the first.  On that account, the grammar of propositional
logic explains how to \emph{construct} a sentence, or perhaps how to
\emph{parse} it into its \dots well, not its meaning, but its
``structure.''  My view, on the other hand, is that the grammar tells
you what a sentence \emph{is}: It is either a sentence letter, or a
negation of a sentence, or a conjuction of an ordered pair of sentences,
etc.  Then, if you like, you might wish to ``linearise'' the sentence as
a string.  Or, given a string, you might wish to parse it back into a
sentence.  But the sentence itself is not a linear object; it has
recursive structure.

Since we have redundant names in this subject perhaps we could remove
the redundancy to make a distinction.  Thus, we could say that a
\emph{proposition} is either an atomic proposition, or a negation of a
proposition, or a conjunction of a proposition, \&c \&c; whereas a
\emph{sentence} is a linearisation of a proposition, so constructed as
to be in one-to-one correspondence with propositions.

Again, perhaps this doesn't matter.


\end{document}

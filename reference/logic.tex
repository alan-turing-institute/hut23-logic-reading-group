% -*- TeX-engine: luatex; fill-column: 72; eval: (auto-fill-mode -1); eval: (visual-fill-column-mode 1); eval: (visual-line-mode 1); eval: (adaptive-wrap-prefix-mode 1) -*-
\documentclass{article}
\usepackage{ttn}
\usepackage{booktabs}
%\usepackage{mathtools}
\usepackage{colonequals} % because unicode-math has broken "::=" somehow
\usepackage{simplebnf}
%\usepackage[raggedrightboxes]{ragged2e}
\usepackage[Ragged, size=footnote, shape=up]{sidenotesplus}
\usepackage[dvipsnames]{xcolor}
%% 
%%\usepackage[style=authoryear]{biblatex}
%%\addbibresource{notes.bib}
%%
%%
%% \defn takes an optional star and an optional argument.
%% If the optional argument is present, it is used for the marginal note; if the argument is missing, the default is the mandatory argument. if the command is starred, no marginal note is made. 
\NewDocumentCommand\defn{sO{#3}m}{\textit{#3}\IfBooleanF{#1}{\sidenote*{\textcolor{MidnightBlue}{#2}}}}
%\newcommand{\defn}[2][#2]{\textit{#2}\sidenote*{#1}}
\newcommand{\isdef}{\stackrel{\text{def}}{=}}
% Relies on euler-math.sty, loaded by ttn.sty
\newcommand{\wff}[1]{\ensuremath{\symscr{#1}}} 
%% 
\title{All we know about logic}
\author{James Geddes}
\date{17 January 2026}
%%
\begin{document}%\maketitle
\maketitle

\section*{Preliminaries}

In the following, ``the book'' means \emph{forall x: Calgary}.

\subsection*{Arguments}

A \defn{sentence} is a statement that could in principle be true or
false.  It is not, for example, a question or an interjection.  (The
book says that ``a \textsc{sentence} [is] a sentence that can be true or
false'' which seems circular.  On the other hand, I have not defined a
``statement.'')

An \defn{argument} is a finite sequence of sentences.  The final sentence is called the \defn[conclusion, premises]{conclusion}; the other sentences are called the \defn*{premises}. 

A \defn{counterexample} to an argument is a situation in which all the
premises are true but the conclusion is false.  An argument is
\defn[valid / invalid]{valid} if it has no counterexamples.  Thus, an
argument is valid if, whenever the premises are all true, then
necessarily the conclusion is true.  An argument that is not valid is
\defn*{invalid}.

The conclusion of a valid argument may be false if its premises are false.  A valid argument whose premises are true is \defn{sound}. If an argument is sound then, indeed, its conclusion may be judged to be true. 

If it is not possible for a sentence to be false, it is \defn[necessary
truth / falsehood]{necessarily true}.  For example: ``The number 7 is
either odd or even.''  If it is not possible for a sentence to be true,
it is \defn*{necessarily false}.  A sentence that is neither necessarily
true nor necessarily false is called \defn{contingent}.

Two sentences are \defn{jointly equivalent} if in all cases either they
are both true or they are both false.  A set of sentences are
\defn[jointly possible / impossible]{jointly possible} if there is a
case in which they are all true; sentences that are not jointly possible
are \defn*{jointly impossible}.

\sidenote*{The book refers to the logic developed in parts II to~IV\ as
  \emph{truth-functional logic}, or TFL\@.  Truth-functional logic is
  also known as \emph{propositional logic} and sometimes
  \emph{sentential logic} (see Appendix~A.1 of the book).  The Stanford
  Encyclopedia of Philosophy calls it propositional logic, so that is
  what I shall do.}
\section*{Propositional logic}

\subsection*{Syntax}
We assume the existence of a countable set of \defn[sentence
letter]{sentence letters}: $A$, $B$, \ldots, $A_1$, $A_2$, \ldots,
$C_{42}$, \ldots.  The symbols $\wedge$, $\vee$, $\to$,
$\leftrightarrow$, and $\neg$ are known as \defn[connectives ($\wedge$,
$\vee$, $\to$, $\leftrightarrow$, and $\neg$)]{connectives}.  A \defn[sentence (or ``proposition'')]{sentence} (of
propositional logic) is defined by the grammar shown in
figure~\ref{fig:bnf-prop}.

\begin{figure}[ht]
  \sidecaption{\label{fig:bnf-prop}The meaning of ``sentence'' in
    propositional logic.  Here \wff{S}, \wff{A}, and \wff{B} are
    arbitrary sentences.  An alternative name for a sentence is a
    \emph{well-formed formula}. Sometimes the symbols $\top$ (``top,'' or
    ``truth'') and $\bot$ (``bottom,'' or ``false'') are also allowed as
    sentences.}
  \centering
  \begin{bnf}
    $\wff{S}$ : Sentence ::=
    | $A$ // $B$ // $C$ // $\dotsb$        : atomic sentence
    |  $\neg \wff{A}$                      : negation
    | ($\wff{A} \wedge \wff{B}$)           : conjuction 
    | ($\wff{A} \vee \wff{B}$)             : disjunction
    | ($\wff{A} \to \wff{B}$)              : conditional
    | ($\wff{A} \leftrightarrow \wff{B}$)  : biconditional
  \end{bnf}
\end{figure}
It is traditional, when writing a sentence, to omit the outer set of
parentheses and to allow the use of other kinds of brackets, such as `['
and `]', wherever parentheses may be used. It is also traditional (but
this is not a tradition followed by the book) to omit many other
parentheses when the result can be disambiguated according to the
following rules:
\begin{enumerate}
\item $\neg$ binds more tightly than $\wedge$ and $\vee$, which bind more tightly
  than $\to$ and $\leftrightarrow$;
\item $\wedge$ and $\vee$ are associative so it doesn't matter how you replace the parentheses in, for example, $P\wedge Q\wedge R$; and
\item $\to$ is taken to be right associative.
\end{enumerate}
Thus $P\wedge Q\to R$ means $(P\wedge Q)\to R$; $P\to Q\to R$ means $P \to (Q \to R)$; and $P\wedge Q \wedge R$ may be read as either $(P\wedge Q)\wedge R$ or $P\wedge (Q\wedge R)$, as you like.



\end{document}

% -*- TeX-engine: luatex; fill-column: 72; eval: (auto-fill-mode -1); eval: (visual-fill-column-mode 1); eval: (visual-line-mode 1); eval: (adaptive-wrap-prefix-mode 1) -*-
\documentclass{article}
\usepackage{ttn}
\usepackage{booktabs}
%%
%\usepackage[raggedrightboxes]{ragged2e}
%\usepackage[Ragged, size=footnote, shape=up]{sidenotesplus}
\usepackage[size=footnote, shape=up]{sidenotesplus}
\usepackage[dvipsnames]{xcolor}
%% 
%%\usepackage[style=authoryear]{biblatex}
%%\addbibresource{notes.bib}
%%
%%
%% \defn takes an optional star and an optional argument.
%% If the optional argument is present, it is used for the marginal note; if the argument is missing, the default is the mandatory argument. if the command is starred, no marginal note is made. 
\NewDocumentCommand\defn{sO{#3}m}{\textit{#3}\IfBooleanF{#1}{\sidenote*{\textcolor{MidnightBlue}{#2}}}}
%\newcommand{\defn}[2][#2]{\textit{#2}\sidenote*{#1}}
\newcommand{\isdef}{\stackrel{\text{def}}{=}}
%% 
\title{All we know about logic}
\author{James Geddes}
\date{17 January 2026}
%%
\begin{document}%\maketitle
\maketitle

\section*{Preliminaries}

\subsection*{Arguments}

A \defn{sentence} is a statement that could in principle be true or
false.  It is not, for example, a question or an interjection.  (The
book says that ``a \textsc{sentence} [is] a sentence that can be true or
false'' which seems circular.  On the other hand, I have not defined a
``statement.'')

An \defn{argument} is a finite sequence of sentences.  The final sentence is called the \defn[conclusion, premises]{conclusion}; the other sentences are called the \defn*{premises}. 

A \defn{counterexample} to an argument is a situation in which all the
premises are true but the conclusion is false.  An argument is
\defn[valid / invalid]{valid} if it has no counterexamples.  Thus, an
argument is valid if, whenever the premises are all true, then
necessarily the conclusion is true.  An argument that is not valid is
\defn*{invalid}.

The conclusion of a valid argument may be false if its premises are false.  A valid argument whose premises are true is \defn{sound}. If an argument is sound then, indeed, its conclusion may be judged to be true. 

If it is not possible for a sentence to be false, it is \defn[necessary
truth / falsehood]{necessarily true}.  For example: ``The number 7 is
either odd or even.''  If it is not possible for a sentence to be true,
it is \defn*{necessarily false}.  A sentence that is neither necessarily
true nor necessarily false is called \defn{contingent}.

Two sentences are \defn{jointly equivalent} if in all cases either they
are both true or they are both false.  A set of sentences are
\defn[jointly possible / impossible]{jointly possible} if there is a
case in which they are all true; sentences that are not jointly possible
are \defn*{jointly impossible}.


\end{document}
